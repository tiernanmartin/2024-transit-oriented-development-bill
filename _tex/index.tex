% Options for packages loaded elsewhere
\PassOptionsToPackage{unicode}{hyperref}
\PassOptionsToPackage{hyphens}{url}
\PassOptionsToPackage{dvipsnames,svgnames,x11names}{xcolor}
%
\documentclass[
  letterpaper,
  DIV=11,
  numbers=noendperiod]{scrartcl}

\usepackage{amsmath,amssymb}
\usepackage{iftex}
\ifPDFTeX
  \usepackage[T1]{fontenc}
  \usepackage[utf8]{inputenc}
  \usepackage{textcomp} % provide euro and other symbols
\else % if luatex or xetex
  \usepackage{unicode-math}
  \defaultfontfeatures{Scale=MatchLowercase}
  \defaultfontfeatures[\rmfamily]{Ligatures=TeX,Scale=1}
\fi
\usepackage{lmodern}
\ifPDFTeX\else  
    % xetex/luatex font selection
\fi
% Use upquote if available, for straight quotes in verbatim environments
\IfFileExists{upquote.sty}{\usepackage{upquote}}{}
\IfFileExists{microtype.sty}{% use microtype if available
  \usepackage[]{microtype}
  \UseMicrotypeSet[protrusion]{basicmath} % disable protrusion for tt fonts
}{}
\makeatletter
\@ifundefined{KOMAClassName}{% if non-KOMA class
  \IfFileExists{parskip.sty}{%
    \usepackage{parskip}
  }{% else
    \setlength{\parindent}{0pt}
    \setlength{\parskip}{6pt plus 2pt minus 1pt}}
}{% if KOMA class
  \KOMAoptions{parskip=half}}
\makeatother
\usepackage{xcolor}
\setlength{\emergencystretch}{3em} % prevent overfull lines
\setcounter{secnumdepth}{5}
% Make \paragraph and \subparagraph free-standing
\ifx\paragraph\undefined\else
  \let\oldparagraph\paragraph
  \renewcommand{\paragraph}[1]{\oldparagraph{#1}\mbox{}}
\fi
\ifx\subparagraph\undefined\else
  \let\oldsubparagraph\subparagraph
  \renewcommand{\subparagraph}[1]{\oldsubparagraph{#1}\mbox{}}
\fi


\providecommand{\tightlist}{%
  \setlength{\itemsep}{0pt}\setlength{\parskip}{0pt}}\usepackage{longtable,booktabs,array}
\usepackage{calc} % for calculating minipage widths
% Correct order of tables after \paragraph or \subparagraph
\usepackage{etoolbox}
\makeatletter
\patchcmd\longtable{\par}{\if@noskipsec\mbox{}\fi\par}{}{}
\makeatother
% Allow footnotes in longtable head/foot
\IfFileExists{footnotehyper.sty}{\usepackage{footnotehyper}}{\usepackage{footnote}}
\makesavenoteenv{longtable}
\usepackage{graphicx}
\makeatletter
\def\maxwidth{\ifdim\Gin@nat@width>\linewidth\linewidth\else\Gin@nat@width\fi}
\def\maxheight{\ifdim\Gin@nat@height>\textheight\textheight\else\Gin@nat@height\fi}
\makeatother
% Scale images if necessary, so that they will not overflow the page
% margins by default, and it is still possible to overwrite the defaults
% using explicit options in \includegraphics[width, height, ...]{}
\setkeys{Gin}{width=\maxwidth,height=\maxheight,keepaspectratio}
% Set default figure placement to htbp
\makeatletter
\def\fps@figure{htbp}
\makeatother
% definitions for citeproc citations
\NewDocumentCommand\citeproctext{}{}
\NewDocumentCommand\citeproc{mm}{%
  \begingroup\def\citeproctext{#2}\cite{#1}\endgroup}
\makeatletter
 % allow citations to break across lines
 \let\@cite@ofmt\@firstofone
 % avoid brackets around text for \cite:
 \def\@biblabel#1{}
 \def\@cite#1#2{{#1\if@tempswa , #2\fi}}
\makeatother
\newlength{\cslhangindent}
\setlength{\cslhangindent}{1.5em}
\newlength{\csllabelwidth}
\setlength{\csllabelwidth}{3em}
\newenvironment{CSLReferences}[2] % #1 hanging-indent, #2 entry-spacing
 {\begin{list}{}{%
  \setlength{\itemindent}{0pt}
  \setlength{\leftmargin}{0pt}
  \setlength{\parsep}{0pt}
  % turn on hanging indent if param 1 is 1
  \ifodd #1
   \setlength{\leftmargin}{\cslhangindent}
   \setlength{\itemindent}{-1\cslhangindent}
  \fi
  % set entry spacing
  \setlength{\itemsep}{#2\baselineskip}}}
 {\end{list}}
\usepackage{calc}
\newcommand{\CSLBlock}[1]{\hfill\break\parbox[t]{\linewidth}{\strut\ignorespaces#1\strut}}
\newcommand{\CSLLeftMargin}[1]{\parbox[t]{\csllabelwidth}{\strut#1\strut}}
\newcommand{\CSLRightInline}[1]{\parbox[t]{\linewidth - \csllabelwidth}{\strut#1\strut}}
\newcommand{\CSLIndent}[1]{\hspace{\cslhangindent}#1}

\KOMAoption{captions}{tableheading}
\makeatletter
\@ifpackageloaded{caption}{}{\usepackage{caption}}
\AtBeginDocument{%
\ifdefined\contentsname
  \renewcommand*\contentsname{Table of contents}
\else
  \newcommand\contentsname{Table of contents}
\fi
\ifdefined\listfigurename
  \renewcommand*\listfigurename{List of Figures}
\else
  \newcommand\listfigurename{List of Figures}
\fi
\ifdefined\listtablename
  \renewcommand*\listtablename{List of Tables}
\else
  \newcommand\listtablename{List of Tables}
\fi
\ifdefined\figurename
  \renewcommand*\figurename{Figure}
\else
  \newcommand\figurename{Figure}
\fi
\ifdefined\tablename
  \renewcommand*\tablename{Table}
\else
  \newcommand\tablename{Table}
\fi
}
\@ifpackageloaded{float}{}{\usepackage{float}}
\floatstyle{ruled}
\@ifundefined{c@chapter}{\newfloat{codelisting}{h}{lop}}{\newfloat{codelisting}{h}{lop}[chapter]}
\floatname{codelisting}{Listing}
\newcommand*\listoflistings{\listof{codelisting}{List of Listings}}
\makeatother
\makeatletter
\makeatother
\makeatletter
\@ifpackageloaded{caption}{}{\usepackage{caption}}
\@ifpackageloaded{subcaption}{}{\usepackage{subcaption}}
\makeatother
\ifLuaTeX
  \usepackage{selnolig}  % disable illegal ligatures
\fi
\usepackage{bookmark}

\IfFileExists{xurl.sty}{\usepackage{xurl}}{} % add URL line breaks if available
\urlstyle{same} % disable monospaced font for URLs
\hypersetup{
  pdftitle={Unlocking Smart Growth: The Effects of Proposed Transit-Oriented Development Laws on the Puget Sound Region},
  pdfauthor={Tiernan Martin},
  pdfkeywords={Transit-Oriented Development, Puget Sound
Region, Washington State 2024 Legislative Session},
  colorlinks=true,
  linkcolor={blue},
  filecolor={Maroon},
  citecolor={Blue},
  urlcolor={Blue},
  pdfcreator={LaTeX via pandoc}}

\title{Unlocking Smart Growth: The Effects of Proposed Transit-Oriented
Development Laws on the Puget Sound Region}
\usepackage{etoolbox}
\makeatletter
\providecommand{\subtitle}[1]{% add subtitle to \maketitle
  \apptocmd{\@title}{\par {\large #1 \par}}{}{}
}
\makeatother
\subtitle{Exploring the Impact of the Community and Transit-Oriented
Housing Development Bills (HB 2160/SB 6024) Across Washington State's
Central Puget Sound Region}
\author{Tiernan Martin}
\date{2024-02-08}

\begin{document}
\maketitle
\begin{abstract}
During the 2024 legislative session in Washington State, two bills were
introduced in both the House and the Senate aimed at promoting community
and transit-oriented housing development. These bills, HB 2160 and SB
6024, propose mandating cities to permit developments of a specific
scale within certain distances from high-capacity transit stops. This
study evaluates the extent to which the proposed increases in
development capacity under these bills exceed current allowances. The
findings indicate a substantial enhancement in development potential for
the majority of areas within walking distance of transit stops.
Specifically, for land that is developable and presently zoned for lower
development capacity than what the bills propose, the average increase
in capacity is projected to be +1.35 in terms of floor area ratio (FAR).
\end{abstract}

\section{Introduction}\label{introduction}

\textsubscript{Source:
\href{https://tiernanmartin.github.io/2024-transit-oriented-development-bill/index-preview.html}{Article
Notebook}}

\phantomsection\label{cell-fig-timeline}
\begin{figure}[H]

\centering{

\includegraphics{index_files/figure-pdf/fig-timeline-1.pdf}

}

\caption{\label{fig-timeline}Timeline of recent earthquakes on La Palma}

\end{figure}%

\textsubscript{Source:
\href{https://tiernanmartin.github.io/2024-transit-oriented-development-bill/index-preview.html}{Article
Notebook}}

\textsubscript{Source:
\href{https://tiernanmartin.github.io/2024-transit-oriented-development-bill/index-preview.html}{Article
Notebook}}

Based on data up to and including 1971, eruptions on La Palma happen
every 79.8 years on average.

Studies of the magma systems feeding the volcano, such as Marrero et al.
(2019), have proposed that there are two main magma reservoirs feeding
the Cumbre Vieja volcano; one in the mantle (30-40km depth) which
charges and in turn feeds a shallower crustal reservoir (10-20km depth).

Eight eruptions have been recorded since the late 1400s
(Figure~\ref{fig-timeline}).

Data and methods are discussed in Section~\ref{sec-data-methods}.

Let \(x\) denote the number of eruptions in a year. Then, \(x\) can be
modeled by a Poisson distribution

\begin{equation}\phantomsection\label{eq-poisson}{
p(x) = \frac{e^{-\lambda} \lambda^{x}}{x !}
}\end{equation}

where \(\lambda\) is the rate of eruptions per year. Using
Equation~\ref{eq-poisson}, the probability of an eruption in the next
\(t\) years can be calculated.

\begin{longtable}[]{@{}ll@{}}
\caption{Recent historic eruptions on La
Palma}\label{tbl-history}\tabularnewline
\toprule\noalign{}
Name & Year \\
\midrule\noalign{}
\endfirsthead
\toprule\noalign{}
Name & Year \\
\midrule\noalign{}
\endhead
\bottomrule\noalign{}
\endlastfoot
Current & 2021 \\
Teneguía & 1971 \\
Nambroque & 1949 \\
El Charco & 1712 \\
Volcán San Antonio & 1677 \\
Volcán San Martin & 1646 \\
Tajuya near El Paso & 1585 \\
Montaña Quemada & 1492 \\
\end{longtable}

Table~\ref{tbl-history} summarises the eruptions recorded since the
colonization of the islands by Europeans in the late 1400s.

\begin{figure}

\centering{

\includegraphics{images/la-palma-map.png}

}

\caption{\label{fig-map}Map of La Palma}

\end{figure}%

La Palma is one of the west most islands in the Volcanic Archipelago of
the Canary Islands (Figure~\ref{fig-map}).

\begin{figure}[H]

\centering{

\includegraphics{index_files/figure-latex/notebooks-explore-earthquakes-fig-spatial-plot-output-1.png}

}

\caption{\label{fig-spatial-plot}Locations of earthquakes on La Palma
since 2017}

\end{figure}%

\textsubscript{Source:
\href{https://tiernanmartin.github.io/2024-transit-oriented-development-bill/notebooks/explore-earthquakes-preview.html\#cell-fig-spatial-plot}{Explore
Earthquakes}}

Figure~\ref{fig-spatial-plot} shows the location of recent Earthquakes
on La Palma.

\section{Data \& Methods}\label{sec-data-methods}

\section{Results}\label{results}

\section{Discussion}\label{discussion}

\section{Conclusion}\label{conclusion}

\section{Acknowledgments}\label{acknowledgments}

\section*{References}\label{references}
\addcontentsline{toc}{section}{References}

\phantomsection\label{refs}
\begin{CSLReferences}{1}{0}
\bibitem[\citeproctext]{ref-marrero2019}
Marrero, José, Alicia García, Manuel Berrocoso, Ángeles Llinares,
Antonio Rodríguez-Losada, and R. Ortiz. 2019. {``Strategies for the
Development of Volcanic Hazard Maps in Monogenetic Volcanic Fields: The
Example of {La} {Palma} ({Canary} {Islands}).''} \emph{Journal of
Applied Volcanology} 8 (July).
\url{https://doi.org/10.1186/s13617-019-0085-5}.

\end{CSLReferences}



\end{document}
